\documentclass{article}
\usepackage[margin=1in]{geometry}

\usepackage{graphicx}
\usepackage[space]{grffile}
\usepackage{latexsym}
\usepackage{amsfonts,amsmath,amssymb}
\usepackage{url}
\usepackage[utf8]{inputenc}
\usepackage{fancyref}
\usepackage{hyperref}
\hypersetup{colorlinks=false,pdfborder={0 0 0},}

\makeatletter
\let\@fnsymbol\@arabic
\makeatother

\title{Gene expression and the role of genetic asssimilation in maize domestication}
\author{
 \small\sfbf{Anne Lorant$^{\ast,}$\thanks{These authors contributed equally to the work}, Sarah Pederson$^{\dag,}$\thanks{These authors contributed equally to the work}, Dolores Piperno$^{\S}$, Matthew Hufford$^{\dag},$\thanks{
Corresponding author:  Department of Ecology, Evolution, and Organismal Biology, Iowa State University, Ames, Iowa 50011, USA. 
    E-mail: \mbox{mhufford@iastate.edu}}, and Jeffrey Ross-Ibarra$^{\ast,\ddag,}$}\thanks{
Corresponding author:  Department of Plant Sciences, University of California, Davis, California 95616, USA. 
    E-mail: \mbox{rossibarra@ucdavis.edu}}\\
\\[0.3cm]
   \small\sf $^{\ast}$Department of Plant Sciences, University of California, Davis, California 95616, USA,\\
   \small\sf $^\dag$Department of Ecology, Evolution, and Organismal Biology, Iowa State University, Ames, Iowa 50011, USA,\\
   \small\sf $^\S$Smithsonian Tropical Research Institute, Box 2072, Balboa, Panama,\\
     \small\sf $^\ddag$Center for Population Biology and Genome Center, University of California, Davis, California 95616, USA\\
}

\date{\today}

\bibliographystyle{plain}
\begin{document}

\maketitle 




\section{Introduction}
Assimilation, Cryptic variability \cite{Lauter2002}, Dolores's previous paper \cite{Piperno_Holst_Winter_McMillan_2014}. 





\section{Methods}
\subsection{Growth Chamber Experiment} %Dolores



\subsection{RNA sequencing} %Anne
For the RNAseq, the total RNA was isolated from the leaf tissue of the plants with
RNeasy mini Kit (Qiagen) following the manufacturer{'}s protocol. Under 400ppm and 265ppm of CO2, respectively 12 and 11 biological replicates were included.  RNA quality and concentration was verified using a Bioanalyzer (Agilent RNA Nano). Sequencing library preparation was performed as previously describe \citep{zhong2011high}. 
Briefly, the mRNA was extracted with Dynabeads oligo(dt)25 (Ambion). After chemical shearing with divalent cations, the First strand synthesis was performed with Random Hexamers Primers (Invitrogen) and SuperScript III (Invitrogen). The second strand cDNA was synthetized with DNApol1 (Thermo Scientific) and RNaseH (New England Biolabs).  
The cDNA fragments were prepared for Illumina sequencing. First, the cDNA fragments were repaired with the End-Repair enzyme mix (New England Biolab). An deoxyadenosine triphosphate was added at each 3{'} ends with the Klenow fragment (New England Biolab). Illumina Trueseq adapters (Affymetrix) were added with the Quick ligase kit (New England Biolab). Between each enzymatic step the cDNA was washed with AMpure beads (Beckman Coulter). Finally the The 23 samples were multiplexed and sequenced in one lane of Hiseq 2500 (UCDavis genome center sequencing facility) for 50 bases single-end reads with an insert size of approximately 300 bases. After demultiplexing, 3.8-8.8 million reads were generated for each sample (Table \ref{tab:readmoms}).

\begin{table}[ht]
\label{tab:readmoms}
\begin{tabular}{l c c} 
Sample & Reads & Maize-like mom \\ \hline 
265\_4B.1.txt & 3814875 & \\
400\_1A.1.txt & 4070011 & X \\ 
265\_3C.1.txt & 4139946 & X \\
265\_3A.1.txt & 4399187 & \\
265\_4C.1.txt & 4746629 & X \\
400\_3C.1.txt & 5029499 & X \\
265\_1B.1.txt & 5031069 & \\
265\_2B.1.txt & 5063618 & X \\
400\_3B.1.txt & 5433424 & \\
265\_3B.1.txt & 5564170 & \\
400\_2B.1.txt & 5812095 & X \\
400\_4A.1.txt & 5857687 & \\
400\_1A\_2.1.txt & 5989185 & \\
265\_2B\_2.1.txt & 6467938 & \\
265\_2A.1.txt & 6732943 & \\
265\_4A.1.txt & 7427625 & \\
400\_2B\_2.1.txt & 7455893 & \\
400\_1B.1.txt & 7570271 & \\
265\_1A\_2.1.txt & 7648528 & \\
400\_2A.1.txt & 7882249 & \\
400\_3A.1.txt & 8630267 & \\
265\_1A.1.txt & 8643790 & X \\
400\_4C.1.tx & 8836575 & X \\  
\end{tabular} 
\caption{Reads}
\end{table}

\subsection{Data Analysis} %JRI

Low quality (base quality <33) bases were trimmed using fastx toolkit v. 0.0.13 (\url(http://hannonlab.cshl.edu/fastx\_toolkit/)), and adapters were subsequently removed using scythe (\url(https://github.com/vsbuffalo/scythe)). Trimmed reads were mapped to the AGPv3.21 of the maize genome using STAR version 2.3.0 \cite{Dobin2013} with default parameters. Read counting was performed with HTseq \cite{Anders2014} using a modified version of the ENSEMBL Zea\_mays.AGPv3.21.gff3 annotation file . Only reads with mapping quality 30 or higher were included in subsequent analyses.

% do we want to check if it makes a difference using lower mapping qualities?

Expression analysis was performed using the EdgeR package \cite(Robinson2013).

%more details needed here on EdgeR analysis once more concrete

\subsection{GO terms}
Methods:
Identifying GO term outliers
GO term analyses were performed on AgriGo \url{bioinfo.cau.edu.cn/agriGO/}.  Significant GO terms were selected for a p-value < 0.005, and further filtered for terms with a false discovery rate <0.005.  This left nothing in low c02 while everything in ambient c02.

In order to understand the biological significance of the genes up or down regulated under low concentration of CO2, a gene ontology (GO) study was performed. 
Since teosinte genes have not been well annotated, the Zea mays ssp. genome has been used to perform the analysis.
A singular enrichment analysis (SEA) on Agrigo (website) has been performed to determine the enriched GO terms in the list of the genes identified on www.maizesequence.org .
The sub-ambient condition produced a down regulation in transcripts associated with basic metabolic and synthetic pathways, with the strongest outliers being basal carbohydrate and nitrogen metabolism.  Even using the lenient false discovery rate (FDR) of 0.005, no GO term categories appear overexpressed under low CO2 conditions.  In contrast, examining the GO terms overexpressed in ambient conditions with the same FDR, many critical biosynthetic and metabolic pathways appear, suggesting an overall down regulation of transcripts for metabolism.  
Our results are consistent with previously findings i.e. significant decrease in biomass and yield under low CO2 for C3 and C4 plants as a result of overall lower metabolism. 





\section{Results}
\subsection{General Expression Results}
\ref{fig:FIGURE_ID}
\subsection{Assimilation}

Genes for which assimilation is important should show:
\begin{itemize}
\item Differential expression (DE) between 400 and 265 (this study)
\item DE between maize and teosinte \cite(Swanson-Wagner2012, Hufford2012) (also cite Lemmon)
\item Evidence of selection during domestication \cite(Hufford2012)
\item Canalization (lack of plastic response) in maize (Dolores's test this summer)
\end{itemize}

\subsection{Cryptic variation}
Genes for which cryptic variation is important should show:
\begin{itemize}
\item Decreased coefficient of variation (CoV) among teosinte populations/lines \cite(Swanson-Wagner2012, Hufford2012)
\item Increased CoV of expression in new conditions (this study)
\item Lower CoV in maize than in teosinte in new conditions (Dolores's test this summer)
\end{itemize}

We want to know:
\begin{itemize}
\item Which genes are in the above category? Any interesting biology there?
\item GO term analysis % AL
\item What proportion of all genes showing DE between maize and teosinte and selection during domestication? Can this tell us something about relative importance?
\end{itemize}

We whould also compare:
\begin{itemize}
\item To lists of stress response genes.
\end{itemize}

\section{Discussion}


Points we will have to discuss: %these have all been brought up by folks I have talked to
\begin{itemize}
\item Adaptation: teosinte has adapted in the intervening 10K years. What we see may not be reflective.
\item Stress: Does this say anything about climate, or only stress-response genes?
\item Timing: How does plasticity work if climate change is slow relative to population genetic processes?
\end{itemize}


\bibliography{bibliography/converted_to_latex.bib}

\end{document}

